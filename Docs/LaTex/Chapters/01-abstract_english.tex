\begin{abstract}
{Recently, deep learning models have had an outstanding performance in tasks such as image classification, anomaly detection and natural language processing. These models thrive when the amount of data is large, which is common nowadays. Coupled with the increasing computing power of GPU, it is possible to solve many of otherwise intractable problems. Nevertheless, the deep learning approach has been mainly used in computer vision while other fields have not adopted it as much. Hence, there is an opportunity to use these models to outperform previous results in the welding field.

In the following work the aim is to use deep learning segmentation models to solve the problem of obtaining relevant features of a Gas Metal Arc Welding (GMAW) process. This is done by segmenting video footage to isolate droplets and then calculating features that are relevant to characterize the process. This problem has been addressed before with computer vision techniques, but the methods used lack automation, and processing large volumes of data becomes unfeasible. Therefore, this thesis' approach allows to achieve results in the segmentation problem itself and the automation process, since results can be obtained faster.

The proposed model is a Fully Convolutional Network approach. Several architectures are considered and compared to then use the best one for further calculations, which by means of supervised training can take a large amount of images and return segmentation masks for each one, isolating the droplets from the background. Furthermore, segmentation masks are used to compute geometric and physical properties. This can be helpful to understand and design the welding process, since it would be possible to map the inputs like current, voltage and shielding gas to a measurable output such as position, area, frequency and velocity of droplets.

A literature review is carried out to understand the problem, how it has been addressed so far and to study the approaches for segmentation using deep learning. Then, data is acquired, consisting of videos of GMAW processes depicting globular and spray transfer which are manually labeled and augmented. Later, Fully Convolutional Network based architectures are trained with the labeled data, namely U-Net, DeconvUnet and MultiResUnet. Finally, the resulting segmentation masks are processed to compute geometric and physical properties.

The main conclusion of this work is that the U-Net based approach can reliably segment droplets within a frame, achieving similar results to previous attempts, but with the benefit of being able to process thousands of images in seconds or minutes. Furthermore, relevant properties were obtained, namely position, velocity, acceleration, perimeter, area, volume and surface tension among others, with values mostly in agreement with literature. Hence this work is a successful step towards automation and a better understanding of GMAW.}
\end{abstract}