En primer lugar quiero agradecer a mi comisión, Vivianna, Patricio y Rubén, sin cuya ayuda difícilmente habría sacado este proyecto adelante. Gracias a Patricio, quien además me recibió de la mejor forma en su laboratorio y que mantuvo el contacto y entusiasmo pese a las circunstancias.

En segundo lugar, doy muchas gracias a mi familia, a mis padres y hermanos quienes me han apoyado todo este tiempo. Su confianza en mi me motiva a seguir adelante y mejorar como profesional y como persona. En particular agradezco a la Laurita, quien ha sido una mejor abuela de lo que podría haber deseado.  

En tercer lugar, gracias a mis amigos del colegio, Patricio y Bastian. A mis amigos de la universidad, a la gente de mecánica: Papo, Cami, Rafa, Noe, Ítalo, Tente, Carlos, Panda, Juampa, Guille, Vale, Arraztio, Matheus, Nati, Yani. Fueron una gran motivación para ir a la U, aunque fuese a perder el tiempo. A los cabros de la sala de magíster, sobre todo a Danilo que es un grande, y los almuerzos de dos horas y media jugando uno.

Especialmente le agradezco a José, quien fue de las primeras personas que conocí al entrar a mecánica. Las mejores experiencias que tuve en la U ocurrieron con y gracias a él, los paseos mecánicos, las vacaciones, las prácticas, los memes. Siempre echaré de menos las veces en que nos íbamos de la U en la madrugada luego de estudiar (o para seguir estudiando) a tu casa en Maipú para no tener que pegarme el pique a mi casa.

Finalmente, gracias a la Pao a quien amo muchísimo y me hace feliz poder ser parte de su vida. Admiro su determinación y agradezco que se ría de mis tallas fomes. Y gracias a su familia, que me ha recibido en su casa, probablemente más de lo que me corresponde.